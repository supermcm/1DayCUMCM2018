\section{问题重述}

CT(Computed Tomography)可以在不破坏样品的情况下,利用样品对射线能量的吸收特性对生物组织和工程材料的样品进行断层成像,由此获取样品内部的结构信息。一种典型的二维CT系统如图1所示,平行入射的X射线垂直于探测器平面,每个探测器单元看成一个接收点,且等距排列。X射线的发射器和探测器相对位置固定不变,整个发射-接收系统绕某固定的旋转中心逆时针旋转180次。对每一个X射线方向,在具有512个等距单元的探测器上测量经位置固定不动的二维待检测介质吸收衰减后的射线能量,并经过增益等处理后得到180组接收信息。

请建立相应的数学模型和算法,解决以下问题:
\begin{enumerate}
\item 在正方形托盘上放置两个均匀固体介质组成的标定模板,模板的几何信息如图所示,相应的数据文件见附件1,其中每一点的数值反映了该点的吸收强度,这里称为“吸收率”。对应于该模板的接收信息见附件2。请根据这一模板及其接收信息,确定CT系统旋转中心在正方形托盘中的位置、探测器单元之间的距离以及该CT系统使用的X射线的180个方向。

\item 附件3是利用上述CT系统得到的某未知介质的接收信息。利用(1)中得到的标定参数,确定该未知介质在正方形托盘中的位置、几何形状和吸收率等信息。另外,请具体给出图3所给的10个位置处的吸收率,相应的数据文件见附件4。
\end{enumerate}

